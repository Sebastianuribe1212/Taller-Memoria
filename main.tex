\documentclass{article}
\usepackage[utf8]{inputenc}
\usepackage[spanish]{babel}
\usepackage{listings}
\usepackage{graphicx}
\graphicspath{ {images/} }
\usepackage{cite}

\begin{document}

\begin{titlepage}
    \begin{center}
        \vspace*{1cm}
            
        \Huge
        \textbf{Nociones de la memoria del computador}
            
        \vspace{0.5cm}
        \LARGE
        Informática II
            
        \vspace{1.5cm}
            
        \textbf{Sebastián Uribe Álvarez}
            
        \vfill
            
        \vspace{0.8cm}
            
        \Large
        Despartamento de Ingeniería Electrónica y Telecomunicaciones\\
        Universidad de Antioquia\\
        Medellín\\
        Septiembre de 2020
            
    \end{center}
\end{titlepage}

\tableofcontents

\section{Sección introductoria}
Este es un proyecto de investigación donde se documentan las respuestas planteadas en el taller "Nociones de la memoria del computador", dicho taller es el primero del curso .

\section{Sección de contenido} \label{contenido}

Desarrollo del taller "Nociones de la memoria delcomputador"
. \cite{refer}


\subsection{Pregunta 1}

1. Defina que es la memoria del computador.
. \cite{refer}
\subsubsection{Desarrollo pregunta 1.}


lo que se le llama memoria de computador 
es el conjunto de modulos que almacenan 
la información, esta ayuda a que los microprocesadores puedan realizar su función de traducir la informaciòn que constantemente llega a ellos.

\vspace{0.3cm}
Estos modulos de memoria son de distintos tipos pero se pueden dividir en dos grupos que los contiene a todos, la memoria volatil y la no volatil, la memoria NO volatil es aquella que almacena los archivos que no queremos perder y que el usuario guarda para su uso en un futuro como por ejemplo el disco duro, y las volatiles son aquellas que se usan para que la información que haya en ellas se almacene por un corto periodo de tiempo, como si aquella información estuviera de paso y se fuera cuando esta ya fuese interpretada por otros procesos del computador un ejemplo de la memoria volatil es la memoria RAM.


\subsection{Pregunta 2}
2. Mencione los tipos de memoria que conoce y haga una pequeña descripción de cada tipo.
. \cite{refer}
\subsection{Desarrollo pregunta 2}



\section{Conclusión} \label{conclulsion}

\bibliographystyle{IEEEtran}
\bibliography{references}

\end{document}
